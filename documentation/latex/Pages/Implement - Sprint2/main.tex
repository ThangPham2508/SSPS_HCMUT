\section{\textcolor{blue}{Hiện thực - Sprint2}} 
\subsection{Các framework sử dụng}
\subsubsection{Front-end}
    $\indent$Với front-end, nhóm tác giả sử dụng \textcolor{blue}{\textit{ReactJS}} để thiết kế layout và trang trí trang web một cách hiện đại, đẹp và tốn ít công sức thực hiện.  \\
    $\indent$ \textcolor{blue}{\textit{ReactJS}} là một thư viện của \textcolor{blue}{\textit{Javascript}}, giúp cho lập trình viên có thể tạo ra các giao diện phức tạp bằng việc định nghĩa các component nhỏ, đơn giản hơn với sự hỗ trợ đến từ ReactJS. Với việc tách một dự án lớn thành những phần nhỏ, ReactJS giúp cho đội dự án có thể phân chia công việc một cách hợp lý, dễ dàng thảo luận và debug nếu có lỗi.\\
\begin{figure}[H]
    \begin{center}
        \includegraphics[width=0.4\textwidth]{Images/Implementation - Sprint2/reactjs.png}
        \caption{ReactJS}
        \label{fig:arch}
    \end{center}
\end{figure}
$\indent$Bởi vì \textcolor{blue}{\textit{ReactJS}} đi theo hướng \textcolor{blue}{\textit{SPA}} (single page application), nói cách khác là ứng dụng sẽ được tải đúng một lần cho toàn bộ, lúc này người dùng sẽ mất thời gian lâu hơn cho việc tải lần đầu, song ở những lần truy cập tới, việc sử dụng trang web sẽ mượt mà hơn như đang thao tác với một ứng dụng di động, tăng trải nghiệm tương tác của người dùng.\\
$\indent$ Ngoài ra, nhóm tác giả sử dụng TailwindCSS - một Framework CSS để thiết kế front-end bởi vì khác với các framework CSS truyền thống như Bootstrap hay Foundation, Tailwind không định nghĩa sẵn các thành phần như nút, biểu mẫu, hay thanh điều hướng. Thay vào đó, Tailwind tập trung vào việc cung cấp các lớp CSS để người dùng có thể xây dựng các thành phần theo ý muốn của mình.
\subsubsection{Back-end}
$\indent$Với back-end, nhóm tác giả sử dụng \textcolor{blue}{\textit{NodeJS}} bởi vì \textcolor{blue}{\textit{NodeJS}}  sử dụng mô hình không đồng bộ (asynchronous) và event-driven, giúp tối ưu hóa sự xử lý đa nhiệm và tăng hiệu suất của ứng dụng. Điều này làm cho nó trở nên hiệu quả trong việc xử lý nhiều yêu cầu đồng thời. \\
\begin{figure}[H]
    \begin{center}
        \includegraphics[width=0.4\textwidth]{Images/Implementation - Sprint2/Nodejs_logo_light.png}
        \caption{NodeJS}
        \label{fig:arch}
    \end{center}
\end{figure}
$\indent$Ngoài ra, nhóm cũng sử dụng \textcolor{blue}{\textit{MongoDB}} để tạo và quản lý cơ sở dữ liệu. Lí do là bởi MongoDB không yêu cầu một schema cố định, cho phép lưu trữ các bản ghi với các trường khác nhau trong cùng một bộ sưu tập (collection). Điều này giúp dễ dàng thay đổi cấu trúc của dữ liệu mà không cần phải cập nhật schema. Và hơn hết MongoDB được sử dụng rộng rãi trong cộng đồng NodeJS. Có nhiều thư viện driver MongoDB cho NodeJS giúp nhóm có thể tương tác với cơ sở dữ liệu MongoDB từ ứng dụng NodeJS một cách thuận tiện.
\begin{figure}[H]
    \begin{center}
        \includegraphics[width=0.4\textwidth]{Images/Implementation - Sprint2/Mongodb.png}
        \caption{MongoDB}
        \label{fig:arch}
    \end{center}
\end{figure}
\subsection{Demo sản phẩm}











$\indent$ Bởi vì nhóm sử dụng bản miễn phí để soạn thảo Latex và dung lượng file quá lớn do đó nhóm chia phải chia nhỏ phần báo cáo MVP-1 và 2 sang một báo cáo khác. Mong thầy (cô) thông cảm.\\
$\indent$ Link truy cập tại: \href{https://drive.google.com/file/d/1lMZ3_EKKinVoFjc2vPUzT27iKlCHLcA9/view?usp=sharing}{Link xem báo cáo phần MVP}

\newpage