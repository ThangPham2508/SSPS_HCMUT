\section{Tổng kết}
$\indent$ Link slide presentation của nhóm: \href{https://drive.google.com/file/d/1wWMBrhd5lnpI_cHHMk5J3UfeLdB0Gkmb/view?usp=sharing}{Xem tại đây}\\
$\indent$ Link video thuyết trình (nhóm đã thuyết trình trực tiếp trên lớp, video này được quay trước để chuẩn bị cho bài báo cáo): \href{https://drive.google.com/file/d/1ALqXauS2n1LKfO6y0Shd7AVJMQImHQW8/view?usp=drive_link}{Xem tại đây} \\
$\indent$ Link video demo sản phẩm: \href{https://drive.google.com/file/d/1DP5tN0lwQjC8eMikUBBkJ7zcTOR_AaeY/view?usp=sharing}{Xem tại đây}\\
$\indent$Trong quá trình hiện thực bài tập lớn, nhóm đã thực hiện các giai đoạn trong quá trình phát triển một phần mềm, từ mô tả và xác định các yêu cầu của dự án đến việc hiện thực giao diện và những chức năng chính của một hệ thống in ấn thông minh tại Trường Đại học Bách khoa TPHCM. Cụ thể hơn, nhóm đã xác định ngữ cảnh, quy trình nghiệp vụ, tầm vực của dự án và các yêu cầu từ người dùng, hệ thống, các yêu cầu chức năng và phi chức năng, sau đó tiếp tục đi vào mô hình hóa hệ thống và thiết kế kiến trúc. \\
$\indent$Để làm rõ các thay đổi của hệ thống trong quá trình vận hành và mô tả cấu trúc cũng như nắm được quy trình xử lý chức năng, nhóm đã vẽ những UML Diagram như Use-case Diagarm, Activity Diagram, Sequence Diagram, Class Diagram, Component Diagram, Architecture (dưới dạng Box-line) và Deployment Diagram. \\
$\indent$Tính đến nay, hệ thống mà nhóm hiện thực cũng đã tương đối hoàn chỉnh. Phần hiện thực của nhóm đã đáp ứng cơ bản những chức năng
cần có của cả khách hàng đó là: Chức năng in ấn (chọn máy in, upload \& config file, mua giấy in...) và của SPSO: Chức năng quản lý máy in (thêm, sửa, xóa, bật, tắt máy in,..), Chức năng quản lý in ấn (chỉnh sửa loại file được in và ngày tặng giấy, số giấy tặng miễn phí). Ngoài ra nhóm cũng đã đáp ứng một số yêu cầu phi chức năng như giao diện thân thiện, dễ sử dụng, xử lý dữ liệu thời gian thực,...\\
$\indent$Bên cạnh những điểm đạt được, nhóm em cũng tự nhận thấy còn nhiều yếu điểm mà nhóm cần phải cải thiện, chủ yếu là trong khâu làm việc nhóm. Vì lượng công việc nhiều, kéo theo việc phân công công việc còn chưa đạt hiệu quả cao khiến nhóm gặp phải nhiều khó khăn, thách thức. Những vấn đề này đã phần nào ảnh hưởng đến chất lượng bài làm của nhóm tại thời điểm đó, thậm chí tạo ra mâu thuẫn giữa các thành viên. Tuy nhiên, nhóm đã không bỏ cuộc mà bình tĩnh tìm hướng giải quyết, tổ chức những lần họp mặt để cùng nhau tìm ra tiếng nói chung, giúp cho việc phối hợp cùng nhau trở nên ăn ý và gắn kết với nhau hơn.\\
$\indent$Nhìn về mặt tích cực hơn thì các thành viên trong nhóm đã có thêm nhiều kinh nghiệm thông qua bài tập lớn lần này. Đó là kinh nghiệm làm việc nhóm, kinh nghiệm thuyết trình, kinh nghiệm trong việc giải quyết tình huống cũng như quản lý thời gian sao cho hiệu quả hơn. Và quan trọng không kém chính là kỹ năng làm việc nhóm để cùng nhau hoàn thiện một phần mềm. Có thể nói, đề tài này đã mang lại cho nhóm rất nhiều kiến thức chuyên ngành quan trọng về quá trình phát triển một phần mềm, từ việc sử dụng UML Diagram sao cho hợp lý cho đến cách dùng github và tận dụng các công cụ như Figma, ReactJS, NodeJS,...hoàn thiện dự án. \\
$\indent$Cuối cùng, nhóm TN01\_04 xin được gửi lời cảm ơn chân thành đến các thầy, cô đã nhiệt tình hướng dẫn nhóm trong suốt quá trình thực hiện bài tập lớn lần này. Nhờ vào việc được các thầy góp ý, trả lời câu hỏi cũng như tổ chức những buổi sửa bài sau mỗi phần, bài làm của nhóm đã được cải thiện và ngày càng trở nên hoàn thiện hơn.
